\documentclass{report}

% Definition : \dfn{}{}
%Frage: \qs{}
%Beweis: \sol{ ...\qed}
%Notiz: \note{}
%Behauptung: \clm{}
%Beispiel: \ex{}
%Theorem: \thm{}{}
%Korollar: \cor{}{}
%Lemma: \mlenma{}{}
%Satz: \mprop{}{}

\input{preamble}
\input{macros}
\input{letterfonts}
\title{Skript Analysis 1 Vorlesung 3} % Sets article title
\author{\textit{Alle Angaben ohne Gewähr}} % Sets authors name
 %Names of file to attach
\usepackage[ngerman]{babel}
\date{\today} % Sets date for publication as date compiled
\setcounter{section}{1}


\begin{document}

\maketitle

\newpage

\section*{Fortsetzung}
\sol{
 Kommutativität $\cap$\\
 \begin{align*}
     A\cap B &= \{x|x\in A \land x\in B\}\\
     \, &= \{x|x\in B \land c\in A\}\\
     \, &=B\cap A
 \end{align*}
 \qed
}\\
\sol{
\begin{align*}
    x\in A \cup (B\cap C) &\iff x\in A \land x\in B\cap C\\
    \, &\iff x\in A\lor (x\in B\land x\in C) \\
    \, &\iff (x\in A \lor x\in B)\land (c\in A \lor x\in C)\\
    \, &\iff x\in A\cup B \land x\in A\cup B\\
    \, &\iff (A\cup B) \cap (A\cup C)\\
\end{align*}
    \qed
}\\
Die restlichen Beweise sind ähnlich


\dfn{Mengenfamilien}{
Sei $J$ beliebige Menge $J\neq \emptyset$\\
Eine Familie von Mengen (Mengenfamilie) ist gegeben durch $A_j$ fpr jeden $j\in J$\\
Schreibe:$$\{A_j\}_{j\in J}$$
}
\dfn{Schnitt und Vereinigungsmengen}{
Es kommt öfters vor, dass man eine Menge $I$ gegeben hat (Indexmenge genannt) und jedem Element $i \in I$ der Indexmenge wird eine Menge $A_i$ zugeordnet. So eine Zuordnung nennt man dann auch Mengenfamilie indiziert über $I$. In so einem Fall schreibt man dann auch:
$$\begin{aligned}
& \bigcap_{i \in I} A_i:=\left\{x \in M \mid \forall i \in I: x \in A_i\right\} \\
& \bigcup_{i \in I} A_i:=\left\{x \in M \mid \exists i \in I: x \in A_i\right\}
\end{aligned}$$
}

\dfn{Kartesisches Produkt}{Sind $M$ und $N$ Mengen, und ist $m \in M$ und $n \in N$, so bezeichnet $(m, n)$ das geordnete Paar bestehend aus $m \in M$ und $n \in N$. Zwei solche Paare $\left(m_1, n_1\right)$ und $\left(m_2, n_2\right)$ sind nach Definition genau dann gleich, wenn $m_1=m_2$ und $n_1=n_2$. Man schreibt
$$
M \times N:=\{(x, y) \mid x \in M \wedge y \in N\}
$$
und nennt $M \times N$ das kartesische Produkt von $M$ und $N$.}

\section{Relationen und Äquivalenzrelationen}

\dfn{Relation}{
Relation $R=(A, B, G)$\\
$G \subset A \times B$ ( $G$ ist der Graph von $R=G_R$ )\\
$(a, b) \in G \quad a$ ist $R$-verwandt zu $a R b$\\
$$
\begin{aligned}
& R_1=\left(A_1, B_1, G_1\right) \\
& R_2=\left(A_2, B_2, G_2\right) \\
& R_1=R_2 \Leftrightarrow A_1=A_2 \wedge B_1=B_2 \wedge G_1=G_2
\end{aligned}
$$}

\dfn{Inverse Relation}{Inverse Relation $R^{-1}$ :
$$
\begin{aligned}
& R^{-1}=\left(B, A, G_{R^{-1}}\right) \\
& G_{R^{-1}}=\left\{(b, a) \mid(a, b) \in G_R\right\}
\end{aligned}
$$}

\dfn{Äquivalenzrelation}{
Sei $R=(A, A, G)$ eine Relation, diese Relation heißt Äquivalenzrelation wenn gilt:
$$R \text{ ist reflexiv: } \forall a \in A: a R a \quad(\forall a \in A:(a, a) \in G)$$
$$R \text{ ist symetrisch: } \forall a_1, a_2 \in A: a_1 R a_2 \Leftrightarrow a_2 R a_1$$
$$R  \text{ ist transitiv: } \forall a_1, a_2, a_3 \in A: a_1 R a_2 \wedge a_2 R a_3 \Rightarrow a_1 R a_3$$\\
$\text { Ist } a_1 R a_2\left(\left(a_1, a_2\right) \in G\right) \text { so nennt man } a_1 \text { äquivalent zu } a_2 \text { bezüglich } R$
}
\end{document}
