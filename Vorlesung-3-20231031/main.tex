\documentclass{report}

\input{preamble}
\input{macros}
\input{letterfonts}
\title{Skript Analysis 1 Vorlesung 1} % Sets article title
\author{\textit{Alle Angaben ohne Gewähr}} % Sets authors name
 %Names of file to attach
\usepackage[ngerman]{babel}
\date{\today} % Sets date for publication as date compiled
\setcounter{section}{0}

\begin{document}

\maketitle

\newpage

\section{Beispiele für Nutzung}

\dfn{Bijektive Funktion}{
Seien $X$ und $Y$ Mengen und sei $f$ eine Abbildung oder eine Funktion, die von $X$ nach $Y$ abbildet, also $f: X \rightarrow Y$. Dann heißt $f$ bijektiv, wenn für alle $y \in Y$ genau ein $x \in X$ mit $f(x)=y$ existiert, formal:
$$
\forall y \in Y: \exists ! x \in X: \quad f(x)=y
$$}

\qs{Die Fragen aller Fragen}{Ist das die Frage?}
\sol{Alle Beweise sind trivial. \qed}
\nt{Es sei $\mathscr{A}(x)$ eine Aussageform. Dann gelten die folgenden Regeln:
$$
\begin{aligned}
& \neg(\forall x: \mathscr{A}(x)) \Longleftrightarrow(\exists x: \neg \mathscr{A}(x)), \\
& \neg(\exists x: \mathscr{A}(x)) \Longleftrightarrow(\forall x: \neg \mathscr{A}(x)) .
\end{aligned}
$$}

\clm{Analysis}{}{Analysis macht spaß!!!}
\ex{}{Die Aussage $\forall x \in \mathbb{N}: x^2=4$ ist falsch, aber die Aussage $\exists x \in \mathbb{N}: x^2=4$ ist wahr.
Bei Aussageformen mit mehreren freien Variablen führt das Anwenden eines Quantors wieder zu einer Aussageform, und erst wenn alle Variablen quantifiziert sind, erhält man eine Aussage.}
\thm{}{}
\cor{}{Durch den oben durchgeführten Beweis könenn wir zeigen...}
\mlenma{}{Angenommen $\vec{v_1}, \dots, \vec{v_n} \in \RR[n]$ ist ein Unterraum von $\RR^n$.}
\mprop{}{Für alle $z, w \in \mathbb{C}$ gilt:\\
1. $\overline{(\bar{z})}=z$,\\
2. $\overline{z+w}=\bar{z}+\bar{w}$,\\
3. $\overline{z w}=\bar{z} \cdot \bar{w}$,\\
4. Ist $z=x+y i$ mit $x, y \in \mathbb{R}$, so ist $z \bar{z}=x^2+y^2$.\\
5. $\Re e(z)=\frac{1}{2}(z+\bar{z}), \Im m(z)=\frac{1}{2 i}(z-\bar{z})$.}
\end{document}
