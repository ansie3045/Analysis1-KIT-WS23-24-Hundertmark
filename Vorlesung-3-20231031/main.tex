\documentclass{report}

% Definition : \dfn{}{}
%Frage: \qs{}
%Beweis: \sol{ ...\qed}
%Notiz: \note{}
%Behauptung: \clm{}
%Beispiel: \ex{}
%Theorem: \thm{}{}
%Korollar: \cor{}{}
%Lemma: \mlenma{}{}
%Satz: \mprop{}{}

\input{preamble}
\input{macros}
\input{letterfonts}
\title{Skript Analysis 1 Vorlesung 3} % Sets article title
\author{\textit{Alle Angaben ohne Gewähr}} % Sets authors name
 %Names of file to attach
\usepackage[ngerman]{babel}
\date{\today} % Sets date for publication as date compiled
\setcounter{section}{1}


\begin{document}

\maketitle

\newpage

\section*{Fortsetzung}
\sol{
 Kommutativität $\cap$\\
 \begin{align*}
     A\cap B &= \{x|x\in A \land x\in B\}\\
     \, &= \{x|x\in B \land c\in A\}\\
     \, &=B\cap A
 \end{align*}
 \qed
}\\
\sol{
\begin{align*}
    x\in A \cup (B\cap C) &\iff x\in A \land x\in B\cap C\\
    \, &\iff x\in A\lor (x\in B\land x\in C) \\
    \, &\iff (x\in A \lor x\in B)\land (c\in A \lor x\in C)\\
    \, &\iff x\in A\cup B \land x\in A\cup B\\
    \, &\iff (A\cup B) \cap (A\cup C)\\
\end{align*}
    \qed
}\\
Die restlichen Beweise sind ähnlich


\dfn{Mengenfamilien}{
Sei $J$ beliebige Menge $J\neq \emptyset$\\
Eine Familie von Mengen (Mengenfamilie) ist gegeben durch $A_j$ fpr jeden $j\in J$\\
Schreibe:$$\{A_j\}_{j\in J}$$
}
\dfn{Schnitt und Vereinigungsmengen}{

}
\end{document}
