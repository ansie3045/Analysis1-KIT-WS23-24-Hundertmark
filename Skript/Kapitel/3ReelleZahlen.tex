\documentclass[../main.tex]{subfiles}
\begin{document}
Es gibt eine Menge \mathbb{R}, genannt reelle Zahlen, die 3 Gruppen von Axiomen erfüllt.
\begin{enumerate}
    \item algebraische Axiome
    \item Anordungsaxiome
    \item Das Vollständigkeitsaxiom
\end{enumerate}
\section{algebraische Axiome}
In \mathbb{R} gibt es zwei Operatoren: Addition $"+"$ und Multiplikation $"\cdot"$\\
$a,b \in \mathbb{R}$ $a+b\in \mathbb{R}$ "Summe", $a\codt b\in \mathbb{R}$ "Produkt"\\
Mit:
\begin{enumerate}
    \item $(a+b)+c=a+(b+c)$ Assoziativgesetz
    \item $a+b=b+a$ Kommutativgesetz
    \item Es gibt genau eine Zahl geannat Null, geschrieben 0, mit $a+0=a\forall a\in \mathbb{R}$
    \item $\forall a\in \mathbb{R}: \exists ! b\in \mathbb{R} : a+b=0$\\
    Schreiben $b=-a$, zu $a $ negatives Element
    \item $(ab)c=a(bc)$
    \item $ab=ba$
    \item Es gibt genau eine Zahl genannt Eins, geschrieben 1, die von Null verschieden ist, mit $a1=a \forall a\in \mathbb{R}$
    \item $\forall a\in \R : a\neq 0$ gibt es eine eindeutige $b\neq 0$, $ab=1$, schreiben $b=a^{-1}=\frac{1}{a}$
    \item $a(b+c)=ab+ac$ Distributivgesetz
\end{enumerate}
!Jede Menge $K$, welche 1.-9. erfüllt, heißt Körper!
Notation:\\
$a-b=a+(-b)$ Differenz\\
$\frac{a}{b} = a/b= a\cdot b^{-1}=b^{-1}\cdot a$ Quotient

\mprop{Abgeleitete Regeln}{Es gilt:\\
\begin{enumerate}
    \setcounter{enumi}{10}
    \item $-(-a)=a \+ (-a)+(-b)=-(a+b)$\\
    $(a^{-1^{-1}}=a \+ a{-1}b^{-1}=(ab)^{-1}$, falls $a,b\neq 0$\\
    $a\cdot 0=0$ (0 ist Monster)\\
    $a(-b)=-(ab) \+ (-a)(-b)=ab$ (Insb. $-a=(-1)a$)\\
    $a(b-c)=ab-ac$
    \item Ist $ab=0$, so ist min. eine der Zahlen $a$ oder $b$ gleich $0$
\end{enumerate}
}
\sol{Zeigen a\cot 0 =0\\
    $$a\cdot 0+a\cdot 0 &=a(0+0)\\
    &=a\cdot 0\\
    addiere &-(a\codt 0\\
    \implies (a0+a0)+(-a0)=a0+(-a0)=0\\
    =a0+(a0+(-a0))=a0+0=a0\\
    \implies a0=0$$}
    
Da $a0=0$\\
$\implies 0=a((b+(-b))=ab+a(-b)$\\
$\implies -(ab)=a(-b)$$

\sol{von 11.\\
Sei $ab=0$\\
Ist $a\neq0 \implies b=1b=(a^{-1}a)b=a^{-1}(ab)=a^{-1}=0$}

\end{document}