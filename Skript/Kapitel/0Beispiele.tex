\documentclass[../main.tex]{subfiles}
\begin{document}
Die Natürlichen Zahlen $\mathbb{N}$ sind alle Zählzahlen ($1,2,3,4,5,...$) \\
$$n\in \mathbb{N} \text{ ist gerade mit } n=2k\\$$
$$n\in \mathbb{N} \text{ ist ungerade mit } n=2k-1$$
\clm{}{}{Für $n\in \mathbb{N}$ ist $n$ gerade, folgt aus $n^2$ gerade\\\\}
\sol{ Direkt:\\
\begin{align*}
     n &= 2k \\
    \implies n^2 &= 4k^2 \\
    \, &= 2 \cdot \underbrace{(2k^2)}_{\in \mathbb{N}} \\
    &\implies n^2 \text{ ist gerade}
\end{align*}\qed}

\clm{}{}{aus $n^2$ gerade folgt $n$ gerade. Diese Aussage zu treffen ist schwierig, da man mit Wurzeln hantieren muss. Was allerdings einfach zu beweisen ist, ist Behauptung 3}

\clm{}{}{Aus $n$ ungerade folgt $n^2$ ungerade}
\sol{durch Kontraposition\\
Annahme: $n=2k-1$ mit $k\in \mathbb{N}$
\begin{align*}
    n^2 &= (2k-1)^2 \\
    \, &= 4k^2 - 4k + 1 \\
    \, &= 2(2k^2 - 2k) + 1 \\
    \, &= 2(2k(k-1) + 1 \\
    \, &= 2\underbrace{(2k(k-1)+1)}_{\in \mathbb{N}} - 1 \\
    &\implies n^2 \text{ ist ungerade}
\end{align*}\qed}

\qs{Was hat Beh. 2 mit Beh. 3 zu tun?}{
p="$n$ ist gerade", q="$n^2$ ist gerade"\\
Beh. 2: Aus q folgt p\\
Beh. 3 Aus nicht p folgt nicht q \textbf{(Kontraposition)}
$$A\implies B \Leftrightarrow \lnot B \implies \lnot A$$
}

\clm{}{}{$\sqrt{2}$ ist irrational}
\sol{Annahme: $\sqrt{2}\in \mathbb{Q}$\\
            $$\text{D.h.} \sqrt{2}=\frac{m}{n} \text{mit }m \in \mathbb{Z}, n\in \mathbb{N}$$
            $$\text{D.h. } A:= \{ n\in \mathbb{N} : \exists m \in \mathbb{Z} \text{ mit } \sqrt{2}=\frac{m}{n}\}$$
            $\sqrt{2}$ ist rational genau dann, wenn A  nicht leer ist.\\
            A ist Teilmenge $\mathbb{N}$\\
            ist A nicht leer, so hat A ein kleinstes Element (\textbf{Prinzip des kleinsten Diebes})\\
            D.h. es existiert $n_*\in A$ mit $n \ge n_*$ deshalb in A $\sqrt{2}=\frac{m}{n_*}$\\
            $$m-n_*=\sqrt{2}n_*-n_*=(\sqrt{2}-1)n_*$$
            Es gilt: $1<\sqrt{2}<2$ $\to$ $0<\sqrt{2}-1< 1$\\
            Also folgt $m-n_*$ ist eine ganze Zahl\\
            und $m-n_*=(\sqrt{2}-1)n_*>0$ also ist $m-n_* \in \mathbb{N} \ge 1$\\
            und $m-n_*=(\sqrt{2}-1)n_*< n_*$\\
            Also ist
            $$\sqrt{2}=\frac{m}{n_*} =
            \frac{m(m-n_*)}{n_*(m-n_*)}=
            \frac{m^2-mn_*}{n_*(m-n_*)}$$
            $$=\frac{2n^2-mn_*}{n_*(m-n_*)}$$
            $$=\frac{n_*(2n_*-m)}{n_*(m-n_*)}=
            \frac{2n_*-m}{m-n_*}=
            \frac{\Tilde{m}}{m-n_*}$$
            wobei $\Tilde{m} \in \mathbb{Z}$\\
            D.h. $m-n_* \in A$\\
            \textbf{Widerspruch} zu $n_*$ ist kleinstes Element von A, da $m-n_*<n_*$\qed}
\clm{}{}{Seien $k \in \mathbb{N}$, dann ist $\sqrt{k} \in \mathbb{N}$ oder $\sqrt{k} \in \mathbb{I}$.}
\sol{durch Widerspruch\\
            $$A:=\{n\in\mathbb{N}|\exists m\in\mathbb{Z}: \sqrt{k}=\frac{m}{n} \text{ für } \sqrt{k} \notin \mathbb{N}\}$$
            \begin{enumerate}
                \item $\sqrt{k}>1$, d.h. es gibt ein $\in \mathbb{N}$ mit $l<\sqrt{k}<l+1$
                \item A hat ein kleinstes Element $n_*$
            \end{enumerate}
            \textit{Man müsste eigentlich beweisen, dass für $\forall M\subseteq \mathbb{N}$ ein kleinstes Element existiert}
            
             $$\sqrt{k}=\frac{m}{n_*}$$
             $$\underbrace{m-ln_*}_{\in \mathbb{Z}}=\sqrt{k}n_*-ln_*=\underbrace{(\sqrt{k}-l)}_{>0}n_*>0$$
             $$\implies (m-ln_*)\in \mathbb{N}$$
             $$m-ln_*=\underbrace{(\sqrt{k}-l)}_{<1}n_*<1n_*=n_*$$
             Also gilt:
            \begin{align*}
               \sqrt{k}=\frac{m}{n_*}&=\, \frac{m(m-ln_*)}{n_*(m-ln_*)}\\
               \,&=\frac{m^2-lmn_*}{n_*(m-ln_*)}\\
               \, &=\frac{kn_*^2-lmn_*}{n_*(m-ln_*)}\\
               \, &=\frac{\overbrace{kn_*-lm}^{\in \mathbb{Z}}}{\underbrace{m-ln_*}_{\in \mathbb{N}}}
            \end{align*}
            $$\implies m-ln_*\in A \text{ Widerspruch, da } m-ln_*<n_*.$$\qed}
\end{document}