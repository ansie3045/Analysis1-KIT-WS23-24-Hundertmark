\documentclass[../main.tex]{subfiles}
\begin{document}

\dfn{Relation}{
Relation $R=(A, B, G)$\\
$G \subset A \times B$ ( $G$ ist der Graph von $R=G_R$ )\\
$(a, b) \in G \quad a$ ist $R$-verwandt zu $a R b$\\
$$
\begin{aligned}
& R_1=\left(A_1, B_1, G_1\right) \\
& R_2=\left(A_2, B_2, G_2\right) \\
& R_1=R_2 \Leftrightarrow A_1=A_2 \wedge B_1=B_2 \wedge G_1=G_2
\end{aligned}
$$}

\dfn{Inverse Relation}{Inverse Relation $R^{-1}$ :
$$
\begin{aligned}
& R^{-1}=\left(B, A, G_{R^{-1}}\right) \\
& G_{R^{-1}}=\left\{(b, a) \mid(a, b) \in G_R\right\}
\end{aligned}
$$}

\ex{}{
A=\{1,2,3,4\}

kleiner Relation $=' < ':=\left(A, A, G_C\right)$
$$
\begin{aligned}
& \text { mit } G_L:=\{(1,2),(1,3),(1,4), (2,3),(2,4),(3,4)\} \\
& a_1<a_2 \Leftrightarrow\left(a_1, a_2\right) \in G_L
\end{aligned}
$$}

\dfn{Äquivalenzrelation}{
Sei $R=(A, A, G)$ eine Relation, diese Relation heißt Äquivalenzrelation wenn gilt:
$$R \text{ ist reflexiv: } \forall a \in A: a R a \quad(\forall a \in A:(a, a) \in G)$$
$$R \text{ ist symetrisch: } \forall a_1, a_2 \in A: a_1 R a_2 \Leftrightarrow a_2 R a_1$$
$$R  \text{ ist transitiv: } \forall a_1, a_2, a_3 \in A: a_1 R a_2 \wedge a_2 R a_3 \Rightarrow a_1 R a_3$$\\
$\text { Ist } a_1 R a_2\left(\left(a_1, a_2\right) \in G\right) \text { so nennt man } a_1 \text { äquivalent zu } a_2 \text { bezüglich } R$
}

\dfn{R Äquivalenzrelation auf A}{
$$
[a]_R:=\{b \in A \mid a R b\}
$$

Für Äquivalenzklassen schreiben wir auch $a \sim_R b$ für $a R b$ oder $a=b$ modulo $R$

}

\nt{Beobachtung: $\forall a \in A$ ist $[a]_R \neq \varnothing$\\
Reflexivität: $a R_a \Rightarrow a \in[a]_R$
$$
\begin{aligned}
a_1, a_2 \in[a]_R & \Rightarrow a_1 v_R a, a_2 \sim_R a \\
& \Rightarrow a_1 v_R a, a \sim_R a_2 \Rightarrow a_1 \sim_R a_2 \text { also } a_1 \in\left[a_2\right]_R
\end{aligned}
$$}

%\pagebreak
\\\\
\clm{R Äquivalenzrelation auf A}{}{Für $a_1, a_2 \in \mathbb{A}$ ist entweder $\left[a_1\right]_R=\left[a_2\right]_R$ oder $[a]_R \cap\left[a_2\right]_R=\varnothing$}

\sol{
    Da $\left[a_1\right]_R,\left[a_2\right]_k \neq \varnothing$ reicht zu zeigen ist $\left[a_1\right]_R \cap\left[a_2\right]_R \neq \varnothing \Rightarrow\left[a_1\right]_R=\left[a_2\right]_R$\\
    Sei $b \in\left[a_1\right]_R \cap\left[a_2\right]_R$\\
    Sei $c \in\left[a_1\right]_R, c \sim_R a_1$\\
    und $b \sim_R a_1 \Rightarrow a_1 \sim_R b$ $\Rightarrow c \sim_R b$\\
    Auch $b \in\left[a_2\right]_R: b \sim_R a_2$ $\Rightarrow c \sim_R a_2$ dh. $c \in\left[a_2\right]_R$\\
    Also ist $\left[a_1\right]_R \subset \left[a_2\right]_R$ \\
    Genauso (Symmetrie) $\left[a_2\right]_R \subset\left[a_1\right]_R$ \qed
}
\\
\cor{}{Ist $R$ Äquivalenzrelation auf $A \neq \varnothing$. Dann sind $a_1, a_2 \in A$ entweder äquivalent oder sie gehören zu disjunkten Äquivalenzrelation.}

\nt{Sei $A \neq \varnothing$. \\
Zerlegung: $F=\left\{A_j\right\}_{j \in J} \quad A_j \subset A$ mit: \\ 
1) $$\forall j \in J: A_j \neq \varnothing$$\\
2) $$\left.j_1, j_2 \in\right], j_1 \neq j_2: A_{j_1} \cap A_{j_2}=\varnothing$$\\
3) $$\bigcup_{j \in J} A_j=A$$}


\end{document}